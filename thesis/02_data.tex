\chapter{Data}

Our main source of data is the Cantus database (\cite{cantus_db}). The database serves as a digital archive of chants, each entry containing
information about its source, liturgical occasion, mode, and others. Work on the project started in the late 1980s, and to date, around 500,000
individual chants from approximately 150 manuscripts have been indexed. Each entry is transcribed manually and undergoes a thorough examination
before publishing (\cite{cantus_lacoste}).

The entire contents of the Cantus database have been scraped previously and are available for download in csv format as the Cantus Corpus (cite???).
As per its description, even the latest version does not contain all of the data in the Cantus database, as this is being updated continuously. However,
for the purposes of this thesis, this does not constitute a problem, as we do not require up-to-date data.

\section{Database fields}

The csv files in Cantus Corpus contain 21 fields (excluding its row number), of which we are only using a subset.

Each entry contains the chant's incipit, which is the first few words of the text. As chants do not have a title, incipit can substitute its role
in contexts where one is needed.

The fields position, sequence, and folio represent the exact location of the original chant in a manuscript from which it was transcribed.

feastid represents the liturgical occasion when the chant was intended to be performed, or, in other words, a feast. Similarly, officeid represents
the liturgical time of the day during which it was sung.

The text and melody of the chant are found in the fulltext and volpiano fields, respectively. Entries can contain both, either, or none of these fields.

\section{Volpiano}

The melodies in the volpiano fields are encoded as strings of alphanumeric characters and dashes. These can be rendered as musical notation using
the volpiano font. Each character represents either a pitch, empty space, or other musical characters, such as a clef.

Volpiano was developed as a research tool optimized for databases and word processors. There are strict rules concerning the transcription, which leads
to all volpiano-encoded melodies having a standardized format. Each transcription begins with a treble clef. Gaps between words are encoded as three
dashes, while two dashes represent gaps between syllables (\cite{volpiano}).

