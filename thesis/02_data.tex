\chapter{Data}

Our main source of data is the Cantus database (\cite{cantus_db}), one of the databases indexed in the Cantus Index. The database serves as a 
digital archive of chants, each entry containing information about its source, liturgical occasion, mode, and others. Work on the project started
in the late 1980s, and to date, around 500,000 individual chants from approximately 150 manuscripts have been indexed. Each entry is transcribed 
manually and undergoes a thorough examination before publishing (\cite{cantus_lacoste}).

We are using a scraped version of the Cantus database released as CantusCorpus (\cite(chant21)). Unlike the Cantus database which is continuously being
updated and is therefore unsuitable for computational study, the corpus is versioned, therefore each version always contains the same data. We are using
version 0.2 released in July 2020 which contains 497071 entries. The corpus is available for download in CSV format.

\section{CSV}

CSV is one of the most common formats for tabular data. The abbreviation stands for \emph{comma-separated values}. As the name suggests, the format
uses commas to separate columns (although other separators, such as a semicolon, can be used as well to allow for simpler parsing in case that the data 
frequently contains commas that would otherwise need to be escaped), while the individual rows are separated by a line break. The data is stored as plaintext,
which makes it easily readable. Parsing CSV files becomes more complicated when the data contains column and row separators inside fields; in that case
quotation marks or esape sign has to be used. There exist many well-designed parsers, one such parser is the Python module simply called \emph{CSV}.

\section{Database fields}

The csv files in Cantus Corpus contain 21 fields (excluding its row number), of which we are only using a subset.

Each entry contains the chant's incipit, which is the first few words of the text. As chants do not have a title, incipit can substitute its role
in contexts where one is needed.

The fields position, sequence, and folio represent the exact location of the original chant in a manuscript from which it was transcribed.

feastid represents the liturgical occasion when the chant was intended to be performed, or, in other words, a feast. Similarly, officeid represents
the liturgical time of the day during which it was sung.

The text and melody of the chant are found in the fulltext and volpiano fields, respectively. Entries can contain both, either, or none of these fields.

\section{Volpiano}

The melodies in the volpiano fields are encoded as strings of alphanumeric characters and dashes. These can be rendered as musical notation using
the volpiano font. Each character represents either a pitch, empty space, or other musical characters, such as a clef.

Volpiano was developed as a research tool optimized for databases and word processors. There are strict rules concerning the transcription, which leads
to all volpiano-encoded melodies having a standardized format. Each transcription begins with a treble clef. Gaps between words are encoded as three
dashes, while two dashes represent gaps between syllables (\cite{volpiano}).

