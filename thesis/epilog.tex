\chapter*{Conclusion}
\addcontentsline{toc}{chapter}{Conclusion}

The outcome of this thesis is an interdisciplinary software tool applying methods from bioinformatics to digital musicology, enabling researchers
to analyze relatively large amounts of data computationally. The software
provides the ability to investigate the origin and evolution of a set of chants. An important implication of this ability is that it facilitates
the discovery of contrafacta (chants with different lyrics but the same melodies) and transpositions (chants whose melodies differ by an interval
in all positions), which is a task that lacked the proper tools until now.

To analyze the chants, we used techniques from bioinformatics, namely multiple sequence alignment algorithms. These techniques have been used by
biologists for a long time to study the same properties - origin and evolution - of biological sequences. As chant melodies and biological sequences
share many properties, and, importantly, MSA algorithms do not require assumptions about chants that we cannot guarantee, these methods are suitable
for studying Gregorian chant.

Our work bridges the gap between traditional musicology,
which relied on the individual ability to draw conclusions from analog sources, and computational analysis. The software is a tool
that will have immediate applications in the basic research of Gregorian chant, such as the study of \emph{Jistebnice Cantionale}.\footnote{
Done in collaboration with the \emph{Old Myths, new Facts} project at the Masaryk Institute of the Czech Academy of Sciences and the Faculty
of Arts of Charles Univeristy.}

Originally, one of the purposes of this work was to create a set of visualizations that would facilitate the quantitative analysis of a set of chants.
However, during the development process it became clear that musicology itself has not yet formalized many of its problems in such a way that 
would make it clear what types of quantitative analysis is actually needed.\footnote{From personal communication with Dr. Jennifer Bain and Dr.
Debra Lacoste.} In fact, the alignment tool proved to be much more useful for immediate
applications\footnote{From personal communication with doc. PhDr. Hana Vlhová-Wörner, Ph.D}. Therefore, our attention shifted to the proper
development of the alignment tool. The visualization part is left for further development, in hopes that the desired use cases crystallize into
a more definite form in the future.

The software continues to be developed in collaboration with the Czech Aca\-demy of Sciences.

\section*{Future work}

Computing multiple sequence alignment opens the door to many other applications. Our software touches upon one
of them by providing the ability to calculate the conservation profile; thhe natural next step is to use the results of the alignment to
obtain and visualize the phylogenetic tree of the set, i.e. their
``family tree'' showing how the individual melodies changed and developed. The tree's visualization could make it easier to see
the relationship between the melodies without the need to study the alignment.

Another possibility is to use the similarity matrix obtained from the alignment to visualize a network of the chants. The clusters
forming in this visualization could show related groups of chants and reveal relationships that were not obvious before.

The opportunities promised by MSA algorithms are vast. We plan to explore more of those that will be determined to be useful for the study of Gregorian
chant, as we continue with the development of the software in collaboration with chant researchers, and we are looking forward to new discoveries.
