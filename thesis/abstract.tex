%%% A template for a simple PDF/A file like a stand-alone abstract of the thesis.

\documentclass[12pt]{report}

\usepackage[a4paper, hmargin=1in, vmargin=1in]{geometry}
\usepackage[a-2u]{pdfx}
\usepackage[utf8]{inputenc}
\usepackage[T1]{fontenc}
\usepackage{lmodern}
\usepackage{textcomp}

\begin{document}

%% Do not forget to edit abstract.xmpdata.

The field of digital musicology studies music using computational methods which enable researchers to process large amounts of data in a short time.
A subset of medieval music, called \emph{Gregorian chant}, is of great interest for many musicologists, as it has had a substantial impact on European culture.
One of the most interesting problems regarding Gregorian chant is its evolution across centuries.
Discovering related chants, and, conversely, unrelated ones, is a necessary step in the handling of the problem, after expert selection of the set of chants to compare.
Computational methods may help with this step, as it requires the processing of large amounts of data.
While there exist large databases of digitalized chants, the field lacks the tools necessary to do this.
This thesis presents a software tool that can help in the discovery of related chants via the employment of bioinformatics methods, namely multiple sequence alignment algorithms.
It enables researchers to align an arbitrary set of chants, thus revealing clusters of related melodies.
Additionally, it facilitates the discovery of contrafacta and transpositions. 
Nevertheless, the tool has some limitations: it is run locally.
Further development is planned as part of the ongoing collaboration with digital musicology researchers from the Czech Academy of Sciences and the Faculty of Arts of Charles University.

\end{document}
