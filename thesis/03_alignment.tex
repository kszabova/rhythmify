\chapter{Sequence alignment}

Sequence alignment, in general, is a task whose purpose is to arrange two or more sequences with a common alphabet to identify similar
and different regions within them. A set of sequences being \emph{aligned} in this case can be understood as each being extended by spaces
in such a way that if they are arranged in a matrix, each sequence occupying one row and each column containing one character, it can be seen
what had to happen for one sequence to change into another on a character-by-character basis: insertion (or deletion), character substitution,
or nothing if the characters in the given position are identical. A good sequence alignment algorithm is one that does not perform
unnecessary insertions (deletions) or substitutions.

The problem of sequence alignment (pairwise and multiple sequence) originated in bioinformatics. Since DNA was first sequenced
in the 1970s, there has been a need to compare various genomes to determine similarity. As organisms mutate and evolve, their
DNA or RNA changes. Aligning their genomes reveals similar and different regions, which facilitates the tracking of these
mutations and makes it possible to determine the order in which they happened.

However, the applications of sequence alignment are not limited to biology. Every task that makes use of determinig the similarity
of some sequences, where the emphasis is put on finding regions where they do not diverge, can make use of the existing methods.

Melody alignment of Gregorian chant can be considered as such. As the tradition spread across Europe, each place changed some of 
the existing melodies by a little, thereby creating new melodies that can change further as they travel through time and space.
This is akin to the mutations in DNA caused by environmental factors. Finding well-conserved regions in many instances of chant
provides great insight into which parts of a melody are unlikely to change, and, on the other hand, which ones tend to vary
a lot. It can also reveal the ancestors of a melody and the path which it traveled to transform into its final form.

In this chapter, we will first give the definition of the problem of sequence alignment. We will mention some important considerations,
as well as theoretical limitations. Then we will provide an overview of the methods developed for bioinformatics that attempt to solve
the problem. Finally, we will show how we applied the existing methods and technologies on Gregorian chant melodies.

\section{The problem of sequence alignment}

Assume that we have an alphabet $\mathcal{A}$ and a character $\sigma$ such that $\sigma \notin \mathcal{A}$. Then let us have a set
of sequences $S = \{s_1, s_2, ..., s_k\}$ with $s_i \in \mathcal{A}^{l_i}$. The output of a sequence alignment algorithm is the set of
aligned sequences $A = \{a_1, a_2, ..., a_k\}$, where $a_k \in (\mathcal{A}\cup\{\sigma\})^L$, $L \geq l_i \:\forall i\in\{1, 2, ..., k\}$.
Additionally, each original sequence $s_i$ is a subsequence of the aligned sequence $a_i$.

Given two aligned sequences $a_i$ and $a_j$ and an index $p \leq L$, we define the following operations:

\begin{itemize}
    \item \emph{Identity}: $(a_i)_k = (a_j)_k$
    \item \emph{Insertion}: $(a_i)_k = \sigma \land (a_j)_k \in \mathcal{A}$
    \item \emph{Deletion}: $(a_i)_k \in \mathcal{A} \land (a_j)_k = \sigma$
    \item \emph{Substitution}: $(a_i)_k \in \mathcal{A} \land (a_j)_k \in \mathcal{A} \land (a_i)_k \neq (a_j)_k$
\end{itemize}

Each of the operations has an associated cost. The cost of substitution can further vary depending on which characters are being
substituted. We can then define the overall cost of the alignment $A$ in different ways, e.g. as the sum of costs over all triples
$(i, j, p) \: \forall i, j \in \{1, 2, ..., k\} \: \forall p \leq L$ or as the sum of costs for unordered pairs $\{i, j\}$ and indices $p$,
in which case insertion and deletion are considered the same operation. The goal of a sequence alignment algorithm is to minimize the cost.
There are other, more complicated ways of defining the cost  function, and the performance of an algorithm is highly dependent on which one it uses.

\subsection{Pairwise and multiple sequence alignment}

Depending on the number of sequences to align, we distinguish between pairwise alignment for pairs of sequences and multiple sequence alignment
for more than two. Despite the similarity in their outcomes, the two problems are fundamentally different from a computational perspective.

Pairwise alignment is relatively easy to solve. The Needleman-Wunsch algorithm, which is a dynamic programming algorithm, can find an optimal
solution in the asymptotic time of $\mathcal{O}(mn)$, where $m$ and $n$ are the respective lengths of the sequences. This means that it is
possible to find an optimal alignment even for longer sequences.

Needleman-Wunsch algorithm can be extended to more than two sequences. However, with each additional sequence, its complexity increases,
and it quickly becomes impractical or even practically impossible to align multiple sequences this way. In fact, it has been proven that multiple sequence
alignment is an NP-complete problem \citep{msa_complexity}. It is therefore necessary to use various heuristics to generate alignments. Current
algorithms do not aim at finding the optimal alignment; instead, they try to produce one that is good enough.

\subsection{Local and global sequence alignment}

There is a distinction to be made between local and global sequence alignment.

The above problem description is the definition of global alignment. Aligning sequences globally means aligning the entire sequences end-to-end.
(This does not mean, however, that there cannot be gaps at the beginning or at the end of the generated alignment.)
All characters from both sequences must be present in the final alignment. Global alignment is used to compare relatively similar
sequences, such as protein homologues or versions of the same chant sung at different points in time.

On the other hand, the goal of local alignment is to find similar regions in divergent sequences, while the rest of the sequences is disregarded.
The output of local alignment algorithms contains only a substring of both sequences. Local alignment is suitable for finding conserved patterns.

Both methods are useful in their own way. Local alignment provides a slightly different insight than global alignment, however, they can
be combined to extract more information. In fact, the best current multiple sequence alignment algorithms use local alignment for pairs
of sequences to generate a better overall global alignment.

\section{Sequence alignment methods}

\subsection{Dynamic programming}

\subsubsection{Needleman-Wunsch algorithm}

The idea of the algorithm is to start with two empty sequences and subsequently add characters from either or both of the given sequences so as
to obtain an optimal alignment in each step. Namely, suppose that we have two sequence prefixes $A$ and $B$ that have already been aligned optimally
and their alignment gives a score of $s$. Furthermore, suppose that the next characters in the sequences are $a$ and $b$, respectively. There are
three possibilities:

\begin{itemize}
    \item We append $a$ and $b$ to the respective prefixes. By doing so, we obtain the aligned sequence prefixes $Aa$ and $Bb$.
        \begin{verbatim}
            Aa
            Bb
        \end{verbatim}
    \item We append $a$ to $A$ and a gap to $B$. This way, we get the aligned prefixes $Aa$ and $B$.
        \begin{verbatim}
            Aa
            B-
        \end{verbatim}
    \item We append a gap to $A$ and $b$ to $B$. Now we have aligned the prefixes $A$ and $Bb$.
        \begin{verbatim}
            A-
            Bb
        \end{verbatim}
\end{itemize}

Each of the possibilities adds a value to the score $s$ depending on what characters were added. To get the optimal alignment, we choose the one
that yields the highest score. We then proceed to the next character, having two optimally aligned prefixes $A'$ and $B'$. This leads to a recursive
algorithm that can be formulated using dynamic programming.

Let us have two input sequences, $A$ and $B$ of lengths $m$ and $n$ with a common alphabet $\mathcal{A}$ and the gap character $\sigma$.
Let us define the scoring function $s$ as

\[ s(a, b) = 
    \begin{cases}
        1  & \text{if } a = b \\
        -1 & \text{if } a = \sigma \lor b = \sigma \\
        -1 & \text{if }  a \neq b
    \end{cases}
\]

The algorithm initializes a matrix $M$ of size $(m+1) * (n+1)$. The rows and columns represent the characters of $A$ and $B$, respectively, except
for the first row and the first column, which represent the beginning of a sequence or an empty sequence. That is to say, the row $M_{i, *}$ represents the
character $A_{i-1}$ for $i \geq 2$ and analogically, the column $M_{*, j}$ represents the character $B_{j-1}$ for $j \geq 2$.

The algorithm iterates over the rows and columns of the matrix. In each step, it calculates the value of a cell $M_{i,j}$, provided that each of the cells $M_{i-1, j}$,
$M_{i, j-1}$ and $M_{i-1, j-1}$ have been filled out, as

\[ M_{i, j} = max
    \begin{cases}
        M_{i-1, j-1} + s(A_{i-1}, B_{j-1}) \\
        M_{i-1, j} + s(A_{i-1}, \sigma)    \\
        M_{i, j-1} + s(\sigma, B_{j-1})
    \end{cases}
\]

In other words, the algorithm either aligns the two characters in positions $i-1$ and $j-1$, or it inserts a gap into sequence $B$, or it inserts a gap into sequence $A$,
and chooses the version which yields the highest score.

The first row and column are apparently special cases, as there is no previous row or column. Therefore, for each cell in the first row or column, there is
only one possible choice of score, which is equivalent to inserting a space.

After filling out the entire matrix, the algorithm then finds the optimal alignment by backtracking in the matrix. It starts in the bottom right
cell of the matrix, which represents both sequences being aligned. In each iteration, it looks at the cells above, to the left and to the top-left
of the current positions and chooses the highest score. If it moves top or left, it means that a gap was inserted. Moving diagonally represents a match
or mismatch. By tracing its way back to the top left corner, the algorithm finds the alignment that yielded the highest score.

Let us show the algorithm on an example. Consider two sequences of nucleotide residues:

\begin{verbatim}
    GATTA
    GCATG
\end{verbatim}

The matrix is initialized without anything filled out.

% create a figure

\begin{verbatim}
          G  C  A  T  G
      
    G 
    A 
    T 
    T 
    A 
\end{verbatim}

We start filling out the matrix in the top left corner. Using the basic scoring scheme (+1 for a match, -1 for everything else), we insert a 0
to the beginning, and, since there is no top cell for the first row and no left cell for the first column, we add -1 to each subsequent cell,
representing a gap insertion.

\begin{verbatim}
          G  C  A  T  G
       0 -1 -2 -3 -4 -5
    G -1
    A -2
    T -3
    T -4
    A -5
\end{verbatim}

The next cell to fill out is the one in the first \verb|G| column and the first \verb|G| row. We have three options:

\begin{itemize}
    \item Move from the top, represents gap insertion for a score of -1. The final score would be $(-1) + (-1) = -2$.
    \item Move diagonally, represents match, giving a score of +1. The score in this case would be $0 + 1 = 1$.
    \item Move from the left, i.e. gap insertion. The score would be $-2$ as in the first case.
\end{itemize}

We choose the highest score, which in this case is a diagonal move representing a match. It follows intuition: we are now aligning
\verb|G| and \verb|G|, matching them seems logical.

\begin{verbatim}
          G  C  A  T  G
       0 -1 -2 -3 -4 -5
    G -1  1
    A -2
    T -3
    T -4
    A -5
\end{verbatim}

Now let us look at the cell in the \verb|C| column and the \verb|G| row. Moving from the top yields -3, moving from the left yields
0 and moving diagonally yields -2, as it is a mismatch in this case. The highest of these scores is 0.

\begin{verbatim}
          G  C  A  T  G
       0 -1 -2 -3 -4 -5
    G -1  1  0
    A -2
    T -3
    T -4
    A -5
\end{verbatim}

We continue filling out the table this way until it is complete.

\begin{verbatim}
          G  C  A  T  G
       0 -1 -2 -3 -4 -5
    G -1  1  0 -1 -2 -3
    A -2  0  0  1  0 -1
    T -3 -1 -1  0  2  1
    T -4 -2 -2 -1  1  1
    A -5 -3 -3 -1  0  0
\end{verbatim}

As we have now calculated the scores for all prefixes, we can use backtracking to find the optimal alignment for the two sequences.
Starting in the bottom right cell, we choose the cell (top, left or top-left) with the highest score. If multiple cells have the same score, we can 
choose either. The different alignments the cells represent are all optimal. The cell that we choose gives us the optimal alignment
of the sequences before the last character was added. Depending on the direction by which we moved, we know whether this last character
was a character of the sequence or a gap.

For example, consider the bottom right corner of the matrix.

\begin{verbatim}
       T  G
    T  1  1
    A  0  0
\end{verbatim}

Starting in the bottom right, we can choose either the top or the top-left cell. Choosing the top-left one, i.e. moving diagonally, means that
the last characters in the alignment were \verb|A| and \verb|G| for the respective sequences. We can find the alignment of the prefixes \verb|GATT|
and \verb|GCAT| by backtracking from the top-left cell. By contrast, choosing the top cell means inserting a gap in the second sequence, and by backtracking
from there we can find the alignment of \verb|GATT| and \verb|GCATG|.

\begin{figure}[h]
    \begin{equation*}
        \begin{matrix}
                &            & G          & C          & A          & T          & G        \\
                & \rn{00}{0} & -1         & -2         & -3         & -4         & -5       \\
            G   & -1         & \rn{11}{1} & 0          & -1         & -2         & -3       \\
            A   & -2         & 0          & \rn{22}{0} & \rn{23}{1} & 0          & -1       \\
            T   & -3         & -1         & -1         & 0          & \rn{34}{2} & 1        \\
            T   & -4         & -2         & -2         & -1         & \rn{44}{1} & 1        \\
            A   & -5         & -3         & 3          & -1         & 0          & \rn{55}{0}
        \end{matrix}
    \end{equation*}

    \begin{tikzpicture}[overlay,remember picture]
        \draw [->] (55) -- (44);
        \draw [->] (44) -- (34);
        \draw [->] (34) -- (23);
        \draw [->] (23) -- (22);
        \draw [->] (22) -- (11);
        \draw [->] (11) -- (00);
    \end{tikzpicture}
\caption{A path representing an optimal alignment.}
\label{alignment_path}
\end{figure}


Figure \ref{alignment_path} shows a path that represents the alignment

\begin{verbatim}
    G A - T T A
    G C A T - G
\end{verbatim}

\subsubsection{Smith-Waterman algorithm}

The algorithm is similar to Needleman-Wunsch algorithm. As its purpose is to find an optimal local alignment, it does not
penalize matches nor gaps. Its purpose is to find regions with the most matches. The only difference from the Needleman-Wunsch 
algorithm is in how new matrix cells are filled out. Namely, when willing out a new cell, we use the formula

\[ M_{i, j} = max
    \begin{cases}
        0   \\
        M_{i-1, j-1} + s(A_{i-1}, B_{j-1}) \\
        M_{i-1, j} + s(A_{i-1}, \sigma)    \\
        M_{i, j-1} + s(\sigma, B_{j-1})
    \end{cases}
\]

In other words, all negative values in what would be the matrix from Needle\-man-Wunsch algorithm are replaced by 0.

\subsection{Progressive methods}

\subsubsection{T-COFFEE}

\subsubsection{MAFFT}

\section{Sequence alignment for chant melodies}
