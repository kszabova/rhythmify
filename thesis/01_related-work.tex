\chapter{Related work}

With the advent of computers, the field of Music Information Retrieval (MIR) emerged. Research in this interdisciplinary field focuses on extracting
information from musical notation using computer science methods, such as signal processing or machine learning. Its applications
vary widely, from recommender systems, to automatic audio transcription, to music generation. MIR encompasses all different kinds of music,
regardless of their location, age, or function. Researchers have developed a multitude of software tools that facilitate music analysis,
irrespective of what type of music it is. One of such toolkits is \emph{music21} \citep{music21}, a Python package able to encode musical notation
as Python objects and perform analysis on large datasets.

The study of plainchant using computational methods has not been done extensively. The main research tool for musicologists in this field
is the Cantus Index \citep{cantus_index}. It is an online index of chants from several different chant databases, providing researchers with 
a common API for all of them. The entries in Cantus Index only contain four data fields: full-text, genre, feast (not required), and Cantus ID,
which is automatically assigned to newly added chants. The tool is also able to search for melodies in the original source and even provides 
search-by-melody functionality. There are ten databases indexed in the catalogue, the largest of which is the Cantus database \citep{cantus_db}.

\cite{chant21} developed \emph{chant21}, a Python package able to convert two standard melodic notations, \emph{volpiano} and \emph{gabc} to
a \emph{music21} object, therefore making it easier to study Gregorian chant computationally. The data they used were 
scraped from Cantus database \citep{cantus_db} and GregoBase \citep{gregobase} and released as CantusCorpus and GregoBaseCorpus, respectively.
Finally, they performed two case studies using the package. In the first one, they confirmed the melodic arch hypothesis \citep{melodic_arch}, 
which had previously only been studied manually. Second, they analyzed the relation between differentiæ and antiphon openings \citep{differentiae}
and found that it differs accross modes.

Some of the computational research into plainchant has been centered on mode classification. \cite{mode_huron} used pitch class profiles to
classify modes. They created a pitch-class distribution for each of the eight modes, and used these classes to classify previously unseen data.
\cite{mode_cornelissen} compared three approaches to mode classification: classical approach, which classifies chants based on
the final pitch, range, and the initial pitch; profile approach, which was largely inspired by \cite{mode_huron}; and distributional approach,
which focuses on the melodic aspect of mode. The authors chose various segmentations and representations of chants and used a tf-idf vector
model to classify mode. The study found that we can accurately classify mode even when we discard all absolute pitch information, the melody
contour contains enough information on its own.

A considerable amount of research has been done into the evaluation of melodic similarity, albeit not for Gregorian chant specifically.
\cite{melodic_similarity} provides an overview of the methods. He mentions edit distance, Markov chains, and geometric measurements as
the most widely used ones. \cite{similarity_plot} used an adapted edit distance metric to calculate the similarity of two melodic sequences
by first calculating the similarity for all segments of each of the sequences and then scaling them by a weight function depending on the
segment length, which yielded them what they call a multi-scale similarity stack. The overall similarity was obtained by averaging its values.
Then they used the MSS stack to create a visualization that takes on the shape of a trapezoid that shows which segments of two sequences
are the most similar.

\cite{similarity_bioinf} argue that methods originally developed for bioinformatics have a great potential to be applied to music. They offer
analogies for bioinformatics concepts found in musicology. For example, they liken DNA and proteins to melodic sequences, homologues (proteins that
have the same ancestor) to song covers, evolution to oral transmission, etc. They claim that despite the similarities, MRI has not leveraged the
full potential of bioinformatics methods. In their article, they focus on modelling melodic similarity using multiple-sequence alignment (MSA)
algorithms, therefore not relying on heuristics, as opposed to previous works. Their results revealed that the MAFFT algorithm yields the best
alignemnt, which can be attributed to the algorithm using gap-free segments as anchor points, therefore partitioning melodies into more meaningful
segments than other algorithms.

The general algorithm for calculating pairwise sequence alignment is the Needleman-Wunsch algorithm \citep{needleman}. It uses dynamic programming
to break down the problem into smaller problems. Given two sequences, it starts aligning them from the beginning. At each point, the algorithm checks
whether the two sequences match in the current position, and if not, whether it will leave the elements mismatched or insert a space. In essence,
all possible alignments are computed and scored and the best one is chosen. The algorithm always yields an optimal alignment, therefore it is
used when the quality of the alignment is important. However, because of its time complexity, it is unsuitable for many applications.

Unlike pairwise sequence alignment, multiple sequence alignment has been shown to be NP-complete \citep{msa_complexity}. As such, there is no practical
way of computing an optimal MSA and we must instead rely on heuristics to obtain a sufficiently good alignment.

\cite{msa_overview} provides an overview of modern multiple sequence alignment algorithms. According to him, the most frequently used algorithms use
the progressive approach, where a guide tree is estimated from unaligned sequences and then pairwise alignment algorithms are used to find
the MSA following the tree. He notes that the scoring methods of the pairwise algorithm are essential. There are two main groups of scoring methods:
matrix-based algorithms, where a substitution matrix is used to determine the cost of replacing one symbol with another, and the consistency-based
methods, which use a collection of global and local alignments to calculate a position-specific substitution matrix. The author claims that
the best methods yield indistinguishable results, except for remote homologs with less than 25\% identity.

T-Coffee \citep{t_coffee} uses the progressive approach described above. It was the first algorithm that used a preprocessed collection of 
alignments to create a library that helps create the guide tree. The library is generated using both global and local pairwise alignments.
Thanks to this approach, T-Coffee minimizes the errors made in the first stages of building the MSA, which is a shortcoming of many previous
algorithms, as these errors tend to persist. They combined precomputed local and global alignments and create a function that assigns a weight to
each pairwise alignment depending on how consistent the pair of residues is with the residue pairs from all other alignments. This process
leads to a significant improvement of the results.

MAFFT \citep{mafft} further improves on other methods by using Fast Fourier transform to identify homologues fast. In addition, the authors
propose a simplified scoring system that reduces CPU time while maintaining its accuracy. The authors' results showed a performance 
100 times better than that of T-Coffee. 

Despite the fact that MSA algorithms have already been used for music analysis, this is the first work that focuses specifically on applying
the methods on Gregorian chant. Gregorian chant is specific by its monophonic nature, which means that there is just one sequence to analyze.
Each chant has also undergone many changes over the centuries, therefore there are many variants of the same chants that can be
researched. These characteristics make Gregorian chant an ideal subject for MSA analysis and application of related bioinformatics methods.
