\chapter{Related work}

With the advent of computers, the field of Music Information Retrieval (MIR) emerged. Research in this interdisciplinary field focuses on extracting
information from musical notation using computer science methods, such as signal processing or machine learning. Its applications
vary widely, from recommender systems, to automatic audio transcription, to music generation. MIR encompasses all different kinds of music,
regardless of their location, age, or function. Researchers have developed a multitude of software tools that facilitate music analysis,
irrespective of what type of music it is. One of such toolkits is \emph{music21} (\cite{music21}), a Python package able to encode musical notation
as Python objects and perform analysis on large datasets.

The study of plainchant using computational methods has not been done extensively. The main research tool for musicologists in this field
is the Cantus Index (\cite{cantus_index}). It is an online index of chants from several different chant databases, providing researchers with 
a common API for all of them.

\cite{chant21} developed \emph{chant21}, a Python package able to convert two standard melodic notations, \emph{volpiano} and \emph{gabc} to
a \emph{music21} object, therefore making it easier to study Gregorian chant computationally. The data they used were 
scraped from Cantus database (\cite{cantus_db}) and GregoBase (\cite{gregobase}) and released as CantusCorpus and GregoBaseCorpus, respectively.
Finally, they performed two case studies using the package. In the first one, they confirmed the melodic arch hypothesis (\cite{melodic_arch}), 
which had previously only been studied manually. Second, they analyzed the relation between differentiæ and antiphon openings (\cite{differentiae})
and found that it differs accross modes.

Some of the computational research into plainchant has been centered on mode classification. \cite{mode_huron} used pitch class profiles to
classify modes. They created a pitch-class distribution for each of the eight modes, and used these classes to classify previously unseen data.
\cite{mode_cornelissen} compared three approaches to mode classification: classical approach, which classifies chants based on
the final pitch, range, and the initial pitch; profile approach, which was largely inspired by \cite{mode_huron}; and distributional approach,
which focuses on the melodic aspect of mode. The authors chose various segmentations and representations of chants and used a tf-idf vector
model to classify mode. The study found that we can accurately classify mode even when we discard all absolute pitch information, the melody
contour contains enough information on its own.

A considerable amount of research has been done into the evaluation of melodic similarity, albeit not for Gregorian chant specifically.
\cite{melodic_similarity} provides an overview of the methods. He mentions edit distance, Markov chains, and geometric measurements as
the most widely used ones. \cite{similarity_plot} used an adapted edit distance metric to calculate the similarity of two melodic sequences
by first calculating the similarity for all segments of each of the sequences and then scaling them by a weight function depending on the
segment length, which yielded them what they call a multi-scale similarity stack. The overall similarity was obtained by averaging its values.
Then they used the MSS stack to create a visualization that takes on the shape of a trapezoid that shows which segments of two sequences
are the most similar.

\cite{similarity_bioinf} argue that methods originally developed for bioinformatics have a great potential to be applied to music. They offer
analogies for bioinformatics concepts found in musicology. For example, they liken DNA and proteins to melodic sequences, homologues (proteins that
have the same ancestor) to song covers, evolution to oral transmission, etc. They claim that despite the similarities, MRI has not leveraged the
full potential of bioinformatics methods. In their article, they focus on modelling melodic similarity using multiple-sequence alignment (MSA)
algorithms, therefore not relying on heuristics, as opposed to previous works. Their results revealed that the MAFFT algorithm yields the best
alignemnt, which can be attributed to the algorithm using gap-free segments as anchor points, therefore partitioning melodies into more meaningful
segments than other algorithms.
